\documentclass[oneside]{article}

\usepackage[utf8]{inputenc}
\usepackage[english]{babel}

% This package offers a versatile interface to add pictures to the document
\usepackage{graphicx}

% This specifies that the page and heading is part of the header of each page
\pagestyle{headings}
\usepackage{blindtext}

% math stuff
\usepackage{amsmath}
\usepackage{amssymb}

% algorithm pseudocode
\usepackage{algorithm}
\usepackage{algpseudocode}

% Specify the T1 font encoding, which is a 8-bit encoding (256 glyphs) ensuring automatic hyphenation amongst other things
% The latex default is OT1 which is a 7-bit encoding (128 glyphs)
\usepackage[T1]{fontenc}

% Geometry package offers an easy interface to adapt the layout of the document, here is use total to specify the dimensions of the body
\usepackage[a4paper, total={16cm, 23cm}]{geometry}

% This package is useful to enter units
\usepackage{siunitx}

\usepackage[
  backend=bibtex,
  style=numeric,       % Numeric citations
  %autocite=superscript, % Superscript in-text numbers
  url=true,            % Ensure URLs are processed
  sortcites=true,
]{biblatex}
\usepackage{url}       % Better URL formatting
\addbibresource{zotero.bib}

% Include URLs in the bibliography entries
\DeclareFieldFormat{url}{\url{#1}}
\renewbibmacro{finentry}{
  \iffieldundef{url}
    {\finentry}
    {\setunit{\addperiod\space}%
     \printfield{url}
     \finentry}}

% Avoid warning: "babel/polyglossia" detected but "csquotes" missing, loading csquotes recommended
\usepackage{csquotes}

% It is recommended to use this package to improve figure placement
% https://tex.stackexchange.com/questions/8625/force-figure-placement-in-text
\usepackage{float}

% Recommended package to align text
\usepackage{ragged2e}

% Create new command for equal sign with less space around them
% from: https://tex.stackexchange.com/questions/365261/how-to-neatly-space-the-equals-sign-when-using-probabilities
\newcommand\myeq{\mkern1.5mu{=}\mkern1.5mu}

% new command for inline code highlighting
% from https://tex.stackexchange.com/questions/286094/insert-code-keywords-inline
\NewDocumentCommand{\codeword}{v}{%
\texttt{\textcolor{black}{#1}}%
}

% Creating hyperlinks when referencing
\usepackage{hyperref}

% Package that makes it easier to reference and insert references as hyperlinks
% It should also be the last package to be imported
% To be able to reference an object (section, figure) I have to give it a label via \label{xyz}
% Then I can reference: 1. Name - \nameref{}, 2. Page - \pageref{}, 
% Some packages to create nice table, according to:
% https://www.tablesgenerator.com
\usepackage{booktabs}

% General Remarks regarding figures
% figures created with inkscape (e.g. gel, microscopy): Serif Font, exported as pdf
% figures created with R/ggplot (e.g. barplot before trimming): Serif Font, exported as pdf

% Set the intend and spacing of paragraphs
\setlength{\parindent}{0ex}
\setlength{\parskip}{1em}

\title{Notes on Archetypal Analysis}
\author{Philipp Sven Lars Schäfer}
\date{April 2025}

\begin{document}

\maketitle

\section{Problem Formulation}

Let $\mathcal{X}=\{\mathbf{x}_1, \ldots, \mathbf{x}_N\}_{n=1}^N$ be a data set consisting of $N$ $D$-dimensional data points, and let $\mathbf{X} \in \mathbb{R}^{N \times D}$ be the matrix where each row is a data point.

In Archetypal Analysis we make two assumptions:
\begin{enumerate}
    \item Each data point is a convex combination of $K$ archetypes;
    \item Each archetype is a convex combination of $N$ data points.
\end{enumerate}

Expressing the first assumption in matrix notation yields
\begin{equation}
\label{eq:first-assumption}
\hat{\mathbf{X}} = \mathbf{A} \mathbf{Z} \quad \text{or} \quad \hat{\mathbf{x}}_n = \mathbf{Z}^T \mathbf{a_n} \text{ for } n = 1, ..., N
\end{equation}
where $\hat{\mathbf{X}} \in \mathbb{R}^{N \times D}$ is the reconstructed data matrix, $\mathbf{Z} \in \mathbb{R}^{K \times D}$ is the matrix of archetypes (i.e. each row is one archetype), and $\mathbf{A} \in \mathbb{R}^{N \times K}$ is a row-stochastic matrix that defines by which archetypes each data point is formed.

Expressing the second assumption in matrix notation yields
\begin{equation}
\label{eq:second-assumption}
\mathbf{Z} = \mathbf{B} \mathbf{X} \quad \text{or} \quad \mathbf{z}_k = \mathbf{X}^T \mathbf{b_k} \text{ for } k = 1, ..., K
\end{equation}
where $\mathbf{B} \in \mathbb{R}^{K \times N}$ is a row-stochastic matrix that defines by which data point each archetype is defined by.

The reconstruction error is most commonly measured using the residual sum of squares (RSS), given by the squared Frobenius norm,

\begin{equation}
\label{eq:rss}
\| \mathbf{X} - \hat{\mathbf{X}} \|_F^2 = \| \mathbf{X} - \mathbf{A} \mathbf{Z} \|_F^2 =  \| \mathbf{X} -  \mathbf{A} \mathbf{B} \mathbf{X} \|_F^2
\end{equation}
which yields the following optimization objective
\begin{equation}
\label{eq:objective}
\begin{aligned}
\mathbf{A}^\star, \mathbf{B}^\star =& \; \underset{\begin{subarray}{c} \mathbf{A} \in \mathbb{R}^{N \times K} \\ 
    \mathbf{B} \in \mathbb{R}^{K \times N} \end{subarray}}{\arg \min} \| \mathbf{X} - \mathbf{A} \mathbf{B} \mathbf{X}\|_F^2 \quad \text{ subject to} \\
& \mathbf{A} \geq 0, \mathbf{A} \mathbf{1}_K = \mathbf{1}_N \\
& \mathbf{B} \geq 0, \mathbf{B} \mathbf{1}_N = \mathbf{1}_K
\end{aligned}
\end{equation}

Introducing the set of row-stochastic non-negative matrices,

\begin{equation}
    F(N, K) := \left\{ \mathbf{A} \in \mathbb{R}^{N \times K} \mid  \mathbf{A} \geq 0 \land \mathbf{A} \mathbf{1}_K = \mathbf{1}_N \right\}
\end{equation}

we can write the objective compactly as:

\begin{equation}
    \begin{aligned}
    \mathbf{A}^\star, \mathbf{B}^\star =& \; \underset{\begin{subarray}{c} \mathbf{A} \in F(N, K) \\ 
        \mathbf{B} \in F(K, N) \end{subarray}}{\arg \min} \| \mathbf{X} - \mathbf{A} \mathbf{B} \mathbf{X}\|_F^2
    \end{aligned}
\end{equation}

\section{Properties of the Objective}

Property 1 (Translation invariance): The minimizers $\mathbf{A}^\star, \mathbf{B}^\star$ of the objective are invariant under row-wise translations of $\mathbf{X}$. Let $\tilde{\mathbf{X}} = \mathbf{X} + \mathbf{1}_N \mathbf{v}^T$ for any $\mathbf{v} \in \mathbb{R}^D$, then

\begin{equation}
    \underset{\begin{subarray}{c} \mathbf{A} \in F(N, K) \\ \mathbf{B} \in F(K, N) \end{subarray}}{\arg \min} \| \tilde{\mathbf{X}} - \mathbf{A} \mathbf{B} \tilde{\mathbf{X}} \|_F^2 
    = 
    \underset{\begin{subarray}{c} \mathbf{A} \in F(N, K) \\ \mathbf{B} \in F(K, N) \end{subarray}}{\arg \min} \| \mathbf{X} - \mathbf{A} \mathbf{B} \mathbf{X}\|_F^2 
\end{equation}

Proof: Let $\mathbf{v} \in \mathbb{R}^{D}$, and let $\tilde{\mathbf{X}} = \mathbf{X} + \mathbf{1}_N \mathbf{v}^T$ be the translated matrix. Then for any feasible $\mathbf{A}, \mathbf{B}$

\begin{equation}
    \begin{aligned}
    \tilde{\mathbf{X}} - \mathbf{A} \mathbf{B} \tilde{\mathbf{X}} 
    &= \left(\mathbf{X} + \mathbf{1}_N \mathbf{v}^T \right) - \mathbf{A} \mathbf{B} \left(\mathbf{X} + \mathbf{1}_N \mathbf{v}^T \right) \\
    &= \mathbf{X} + \mathbf{1}_N \mathbf{v}^T - \mathbf{A} \mathbf{B} \mathbf{X} -  \mathbf{A} \mathbf{B} \mathbf{1}_N \mathbf{v}^T
    \end{aligned}
\end{equation}

Since $\mathbf{B} \mathbf{1}_N = \mathbf{1}_K$ and  $\mathbf{A} \mathbf{1}_K = \mathbf{1}_N$, this simplifies to 

\begin{equation}
    \begin{aligned}
    \tilde{\mathbf{X}} - \mathbf{A} \mathbf{B} \tilde{\mathbf{X}}
    &= \mathbf{X} + \mathbf{1}_N \mathbf{v}^T - \mathbf{A} \mathbf{B} \mathbf{X} - \mathbf{1}_N \mathbf{v}^T \\
    &= \mathbf{X} - \mathbf{A} \mathbf{B} \mathbf{X}
    \end{aligned}
\end{equation}

Therefore, the reconstruction error remains unchanged, and the minimizers $\mathbf{A}^\star, \mathbf{B}^\star$ are invariant under such translations. Thus, the minimizers $\mathbf{A}^\star, \mathbf{B}^\star$ are invariant to centering the data.

Property 2 (Scale invariance): The minimizers $\mathbf{A}^\star, \mathbf{B}^\star$ of the objective are invariant under global scaling of $\mathbf{X}$. Let $\tilde{\mathbf{X}} = \lambda \mathbf{X}$ for any $\lambda \neq 0$, then

\begin{equation}
    \underset{\begin{subarray}{c} \mathbf{A} \in F(N, K) \\ \mathbf{B} \in F(K, N) \end{subarray}}{\arg \min} \| \tilde{\mathbf{X}} - \mathbf{A} \mathbf{B} \tilde{\mathbf{X}} \|_F^2 
    = 
    \underset{\begin{subarray}{c} \mathbf{A} \in F(N, K) \\ \mathbf{B} \in F(K, N) \end{subarray}}{\arg \min} \| \mathbf{X} - \mathbf{A} \mathbf{B} \mathbf{X}\|_F^2 
\end{equation}

Proof: Let $\lambda \neq 0$, and let $\tilde{\mathbf{X}} = \lambda \mathbf{X}$ be the scaled matrix. Then for any feasible $\mathbf{A}, \mathbf{B}$

\begin{equation}
    \begin{aligned}
    \tilde{\mathbf{X}} - \mathbf{A} \mathbf{B} \tilde{\mathbf{X}} 
    &= \lambda \mathbf{X} - \mathbf{A} \mathbf{B} \lambda \mathbf{X} \\
    &= \lambda \left( \mathbf{X} - \mathbf{A} \mathbf{B} \mathbf{X} \right)
    \end{aligned}
\end{equation}

Thus the objective for the scaled matrix is given by

\begin{equation}
    \underset{\begin{subarray}{c} \mathbf{A} \in F(N, K) \\ \mathbf{B} \in F(K, N) \end{subarray}}{\arg \min} \| \tilde{\mathbf{X}} - \mathbf{A} \mathbf{B} \tilde{\mathbf{X}} \|_F^2 = \underset{\begin{subarray}{c} \mathbf{A} \in F(N, K) \\ \mathbf{B} \in F(K, N) \end{subarray}}{\arg \min} \lambda^2 \| \mathbf{X} - \mathbf{A} \mathbf{B} \mathbf{X} \|_F^2
\end{equation}

Since $\lambda \neq 0$, we have $\lambda^2 > 0$, and thus the objective is scaled by a positive constant. Multiplying the objective function by a positive scalar does not affect the location of its minimum, because the ordering of objective values is preserved. In particular, the first-order (stationarity) and second-order (convexity/curvature) necessary conditions for optimality remain unchanged under such scaling. 

Property 3 (unique up to permutation of archetypes / No rotational ambiguity) 

Assuming that for each archetypes there exists one data point that only belong to this archetype

\begin{equation}
    \label{eq:cond1}
    \forall k \in \{1, ..., K\} \exists n \in \{1, ..., N\} \; a_{n k} > 0 \; \land \; a_{a k^\prime} = 0 \forall k^\prime \neq k
\end{equation}

and that for each archetype there exists one data point that is only used to define this archetype and not any other archetype

\begin{equation}
    \label{eq:cond2}
    \forall k \in \{1, ..., K\} \exists n \in \{1, ..., N\} \; z_{k n} > 0 \; \land \; z_{k^\prime n} = 0 \forall k^\prime \neq k
\end{equation}

then the objective does not suffer from rotational ambiguity. 

Note, these conditions mean that both $\mathbf{A}$ and $\mathbf{B}$ have rank $K$ (i.e. $K$ linearly indendent columns / rows).

Let $\mathbf{Q} \in \mathbf{R}^{K \times K}$ be some invertible matrix

\begin{equation}
    \mathbf{A} \mathbf{B} \mathbf{X} = \mathbf{A} \mathbf{Q} \mathbf{Q}^{-1} \mathbf{B} \mathbf{X} = \tilde{\mathbf{A}} \tilde{\mathbf{B}} \mathbf{X}
\end{equation}

Requiring that $\tilde{\mathbf{A}} \in F(N, K)$ and $\tilde{\mathbf{B}} \in F(K, N)$ (i.e. that both $\tilde{\mathbf{A}}$, $\tilde{\mathbf{B}}$ are still row-stochastic), we can derive the following properties that $\mathbf{Q}$ must fullfull.

First, since we require $\tilde{\mathbf{A}} \geq 0$, and $\mathbf{A} \geq 0$, and $\tilde{\mathbf{A}} = \mathbf{A} \mathbf{Q}^{-1}$, and Equation~\eqref{eq:cond1} must hold, we know that $\mathbf{Q} \geq 0$.

Second, since we require $\tilde{\mathbf{B}} \geq 0$, and $\mathbf{B} \geq 0$, and $\tilde{\mathbf{B}} = \mathbf{Q}^{-1} \mathbf{B}$, and Equation~\eqref{eq:cond2} must hold, we know that $\mathbf{Q}^{-1} \geq 0$.

Then, since $\mathbf{Q}$ and $\mathbf{Q}^{-1}$ are non-negative, Lemma 1.1 from \autocite{mincNonnegativeMatrices1988} states that $\mathbf{Q}$ must be a generalized permutation matrix, i.e. there exists some diagonal matrix $\mathbf{D} \in \mathbb{R}^{K \times K}$ and permuation matrix $\mathbf{P} \in \mathbb{R}^{K \times K}$ such that $\mathbf{Q} = \mathbf{D} \mathbf{P}$.

Third, since we require $\tilde{\mathbf{A}} \mathbf{1}_K = \mathbf{1}_N = \mathbf{A} \mathbf{Q} \mathbf{1}_K = \mathbf{1}_N$, we know that $\mathbf{Q} \mathbf{1}_K = \mathbf{1}_K$ and thus $\mathbf{Q} \in F(K, K)$ (i.e. $\mathbf{Q}$ must be a row-stochastic matrix)

Then, this means that $\mathbf{Q}$ must be a permutation matrix since 

\begin{equation}
    \begin{aligned}
        &\mathbf{Q} \mathbf{1}_K = \mathbf{1}_K \\
        \rightarrow &\mathbf{D} \mathbf{P} \mathbf{1}_K = \mathbf{1}_K \\
        \rightarrow &\mathbf{D} \mathbf{1}_K = \mathbf{1}_K \\
        \rightarrow &\mathbf{D} = \mathbf{I}_K
    \end{aligned}
\end{equation}

%Fourth, since we require $\tilde{\mathbf{B}} \mathbf{1}_K = \mathbf{1}_N = \mathbf{Q}^{-1} \mathbf{B} \mathbf{1}_N = \mathbf{1}_K$, we know that $\mathbf{Q}^{-1} \mathbf{1}_K = \mathbf{1}_K$ and thus $\mathbf{Q}^{-1} \in F(K, K)$ (i.e. $\mathbf{Q}^{-1}$ must be a row-stochastic matrix)

Property 4 (Rewrite via convex hull of $\mathbf{Z}$): For some fixed $\mathbf{B} \in F(K, N)$, let the corresponding archetype matrix be $\mathbf{Z} = \mathbf{B} \mathbf{X}$. Then for the optimal $\mathbf{A} \in F(K, N)$, the objective  dfagfa

\begin{equation}
    \begin{aligned}
    \| \mathbf{X} - \mathbf{A} \mathbf{B} \mathbf{X} \|_F^2 
    &= \sum_{n=1}^N \| \mathbf{x}_n - \mathbf{Z}^T \mathbf{a}_n \|_2^2 \\
    &= \sum_{n=1}^N \left\| \mathbf{x}_n - \sum_{k=1}^K a_{nk} \mathbf{z}_k \right\|_2^2
    \end{aligned}
\end{equation}

is equivalent to

\begin{equation}
    \begin{aligned}
    \| \mathbf{X} - \mathbf{A} \mathbf{B} \mathbf{X} \|_F^2 
    &= \sum_{n=1}^N \min_{\mathbf{q} \in \text{conv}(\mathcal{Z})} \left\| \mathbf{x}_n - \mathbf{q} \right\|_2^2
    \end{aligned}
\end{equation}

Thus, for fixed $\mathbf{B}$, the optimal $\mathbf{A}$ assigns each data point to its Euclidean projection onto the convex hull of the archetypes $\mathbf{z}_1, ..., \mathbf{z}_K$. Any point $x_n \in \text{conv}(\mathcal{Z})$ does not contribute to the loss.



\section{Optimization}

While this objective is an Euclidean sum of square clustering problem which have been proven to be NP-hard\autocite{aloiseNPhardnessEuclideanSumofsquares2009}, several practical optimization approaches have been developed that exploit that this objective is biconvex, meaning that it is convex in $\mathbf{A}$ if we fix $\mathbf{B}$ and vice versa. See Section 5 in Cutler \& Breiman (1994) \autocite{cutlerArchetypalAnalysis1994} or Section 2 in Mørup \& Hansen (2012) \autocite{morupArchetypalAnalysisMachine2012} for more details. One way to optimize such a biconvex objective is to initialize $\mathbf{A}$, $\mathbf{B}$, and then alternating between solving the convex optimization problem in one variable fixing the other variable, and vice versa.

\subsection{Gradient of the Objective}

To compute the gradient of the unconstrained objective w.r.t. $\mathbf{A}$ and $\mathbf{B}$, we first rewrite the residual sum of squares (Frobenius norm) in Equation~\eqref{eq:objective} in terms of the trace
\begin{equation}
\begin{aligned}
\operatorname{RSS} &= \| \mathbf{X} - \mathbf{A} \mathbf{B} \mathbf{X}\|_F^2 \\
&= \operatorname{tr} \left( \left( \mathbf{X} - \mathbf{A} \mathbf{B} \mathbf{X} \right)^T \left( \mathbf{X} - \mathbf{A} \mathbf{B} \mathbf{X} \right) \right) \\
&=  \operatorname{tr}(\mathbf{X}^T \mathbf{X}) - \operatorname{tr}(\mathbf{X}^T \mathbf{A} B \mathbf{X}) - \operatorname{tr}(\mathbf{X}^T \mathbf{B}^T \mathbf{A}^T \mathbf{X}) + \operatorname{tr}(\mathbf{X}^T \mathbf{B}^T \mathbf{A}^T \mathbf{A} \mathbf{B} \mathbf{X}) \\
&=  \operatorname{tr}(\mathbf{X}^T \mathbf{X}) - 2 \operatorname{tr}(\mathbf{X}^T \mathbf{A} \mathbf{B} \mathbf{X}) + \operatorname{tr}(\mathbf{X}^T \mathbf{B}^T \mathbf{A}^T \mathbf{A} \mathbf{B} \mathbf{X}) \\
\end{aligned}
\tag{5}
\end{equation}
where we used that for any $\mathbf{G}, \mathbf{H} \in \mathbb{R}^{N \times N}$ it is true that $\operatorname{tr}(\mathbf{G} + \mathbf{H}) = \operatorname{tr}(\mathbf{G})+\operatorname{tr}(\mathbf{H})$ and $\operatorname{tr}(\mathbf{G}^T)=\operatorname{tr}(\mathbf{G})$

Next we will use Equation 101 from the Matrix Cookbook by Petersen and Pedersen (2012) \autocite{petersenMatrixCookbook2012} which states that for any matrices $G, H, J \in \mathbb{R}^{N \times N}$ we have
\begin{equation}
\frac{\partial}{\partial H}\operatorname{tr}(GHJ) = G^T J^T
\end{equation}
and Equation 116 which states that for any matrices $G, H, J \in \mathbb{R}^{N \times N}$ we have
\begin{equation}
\frac{\partial}{\partial H}\operatorname{tr}(G^T H^T J H G) = J^T H G G^T + J H G G^T
\end{equation}
So computing the gradient of the RSS w.r.t. $A$ we have
\begin{equation}
\begin{aligned}
G^{(A)} &= \nabla_A \operatorname{RSS} \\
&=\nabla_A \left[ \operatorname{tr}(X^T X) - 2 \operatorname{tr}(X^T A B X) + \operatorname{tr}(X^T B^T A^T A B X) \right] \\
&= - 2 \nabla_A \operatorname{tr}(\underbrace{X^T}_{G} \underbrace{A}_{H} \underbrace{B X}_{J}) + \nabla_A \operatorname{tr}(\underbrace{(B X)^T}_{G^T} \underbrace{A^T}_{H^T} \underbrace{I}_{J} \underbrace{A}_{H} \underbrace{B X}_{G}) \\
&= - 2 X X^T B^T + \left( I^T A B X X^T B^T + I A B X X^T B^T \right) \\
&= - 2 X X^T B^T + 2 A B X X^T B^T \\
&= 2 \left( A B X X^T B^T - X X^T B^T \right) \\
&= 2 \left( A Z Z^T - X Z^T \right)
\end{aligned}
\end{equation}
Similarly, computing the gradient of the RSS w.r.t. $B$ we have
\begin{equation}
\begin{aligned}
G^{(B)} &= \nabla_B \operatorname{RSS} \\
&= \nabla_A \left[ \operatorname{tr}(X^T X) - 2 \operatorname{tr}(X^T A B X) + \operatorname{tr}(X^T B^T A^T A B X) \right] \\
&= - 2 \nabla_B \operatorname{tr}(\underbrace{X^T A}_{G} \underbrace{B}_{H} \underbrace{X}_{J}) + \nabla_B \operatorname{tr}(\underbrace{X^T}_{G^T} \underbrace{B^T}_{H^T} \underbrace{A^T A}_{J} \underbrace{B}_{H} \underbrace{X}_{G}) \\
& = - 2 A^T X X^T + \left( A^T A B X X^T + A^T A B X X^T \right) \\
& = - 2 A^T X X^T + 2 A^T A B X X^T  \\
& = 2 \left( A^T A B X X^T - A^T X X^T \right)
\end{aligned}
\end{equation}

\subsection{Regularized Nonnegative Least Squares}

Introduced in 1994 by Adele Cutler and Leo Breiman \autocite{cutlerArchetypalAnalysis1994}, this was the first algorithm to solve the archetypal analysis objective in Equation~\eqref{eq:objective}.

\begin{algorithm}
\caption{Archetypal Analysis Algorithm}
\begin{algorithmic}[1]
\State Initialize $\mathbf{B}$ and compute the archetypes $\mathbf{Z} = \mathbf{B} \mathbf{X}$
\While{not converged or maximum number of iterations is reached}
    \For{$n = 1$ to $N$}
        \State Find optimal $\mathbf{a}_n$ by solving the constrained optimization problem:
        $$\mathbf{a}_n = \underset{\mathbf{a}_n \in \mathbb{R}^K}{\arg \min} \|\mathbf{x}_n - \mathbf{Z}^T \mathbf{a}_n\|_2^2 \quad \text{ subject to } \quad \mathbf{a}_n \geq 0, \sum_{k=1}^K a_{nk} = 1$$
    \EndFor
    \State Compute the optimal archetypes $\mathbf{Z}$ given $\mathbf{A}$, i.e.
    $$\mathbf{Z} = \underset{\mathbf{Z} \in \mathbb{R}^{K \times D}}{\arg \min} \| \mathbf{X} - \mathbf{A} \mathbf{Z}\|_F^2$$
    \For{$k = 1$ to $K$}
        \State Find optimal $\mathbf{b}_k$ by solving the constrained optimization problem:
        $$\mathbf{b}_k = \underset{\mathbf{b}_k \in \mathbb{R}^N}{\arg \min} \|\mathbf{z}_k - \mathbf{X}^T \mathbf{b}_k\|_2^2 \quad \text{ subject to } \quad \mathbf{b}_k \geq 0, \sum_{n=1}^N b_{kn} = 1$$
    \EndFor
    \State Compute the archetypes given $\mathbf{B}$, i.e. $\mathbf{Z}=\mathbf{B}\mathbf{X}$
\EndWhile
\State \Return $\mathbf{A}, \mathbf{B}, \mathbf{Z}$
\end{algorithmic}
\end{algorithm}

The authors originally proposed to solve the constrained optimization problems using a Nonnegative Least Squares Problem (NNLS) solver and enforcing the convexity constraints using a penalty term with regularization parameter $\lambda$, i.e.
\begin{equation}
\begin{aligned}
\mathbf{a}_n &= \underset{a_n \in \mathbb{R}^K}{\arg \min} \|x_n - Z^T \mathbf{a}_n\|_2^2 + \lambda \| \mathbf{1}_K - \mathbf{a}_n \|_2^2 \quad \text{ subject to } \quad \mathbf{a}_n \geq 0 \\
&= \underset{\mathbf{a}_n \in \mathbb{R}^K}{\arg \min} \left\| \begin{bmatrix} \mathbf{x}_n \\ \lambda \end{bmatrix} -  \begin{bmatrix} \mathbf{Z}^T \\ \lambda \mathbf{1}_K^T \end{bmatrix} \mathbf{a}_n \right\|_2^2
\end{aligned}
\end{equation}

Equivalently, for $\mathbf{B}$ we have
\begin{equation}
\begin{aligned}
\mathbf{b}_k &= \underset{b_k \in \mathbb{R}^N}{\arg \min} \|\mathbf{z}_k - \mathbf{X}^T \mathbf{b}_k\|_2^2 + \lambda \| \mathbf{1}_N - \mathbf{b}_k \|_2^2 \quad \text{ subject to } \quad \mathbf{b}_k \geq 0 \\
&= \underset{\mathbf{a}_n \in \mathbb{R}^K}{\arg \min} \left\| \begin{bmatrix} \mathbf{z}_k \\ \lambda \end{bmatrix} -  \begin{bmatrix} X^T \\ \lambda \mathbf{1}_N^T \end{bmatrix} b_k \right\|_2^2
\end{aligned}
\end{equation}

\subsection{Principal Convex Hull Algorithm (PCHA)}

Inspired by the projected gradient method for NMF \autocite{linProjectedGradientMethods2007} and normalization invariance approach introduced for NMF \autocite{eggertSparseCodingNMF2004}, the PCHA algorithm was introduced by Morten Mørup and Lars Kai Hansen in 2012 to solve the archetypal analysis objective. 

The idea is to use a projected gradient algorithm to solve the objective in Equation~\eqref{eq:objective}.

First, we recast the optimization problem in terms of the l1-normalization invariant variables $\tilde{a}_n$ and $\tilde{b}_k$ (called invariant because these variables won't change if one applies l1-normalization)
\begin{equation}
\tilde{a}_{nk} = \frac{a_{nk}}{\sum_{k''=1}^K a_{nk''}}, \quad \tilde{b}_{kn} = \frac{b_{kn}}{\sum_{n''=1}^N b_{kn''}}
\end{equation}
Then the gradient of the RSS wrt to $a_{n}$ is obtained using the chain rule which yields
\begin{equation}
\begin{aligned}
\frac{\partial \text{RSS}}{\partial a_n} &= \frac{\partial \text{RSS}}{\partial \tilde{a}_n} \frac{\partial  \tilde{a}_n}{\partial a_n} \\
&= \left( \tilde{g}^{(A)}_n \right)^T \left( \frac{\left( \sum_{k''=1}^K a_{nk''} \right) \mathbf{I}_K - a_n \mathbf{1}_K^T}{\left( \sum_{k''=1}^K a_{nk''} \right)^2} \right) \\
&= \frac{\left( \sum_{k''=1}^K a_{nk''} \right) \left( \tilde{g}^{(A)}_n \right)^T \mathbf{I}_K - \left( \tilde{g}^{(A)}_n \right)^T a_n \mathbf{1}_K^T}{\left( \sum_{k''=1}^K a_{nk''} \right)^2}
\end{aligned}
\end{equation}
So for a single element we have
\begin{equation}
\begin{aligned}
\frac{\partial \text{RSS}}{\partial a_{nk}} &= \frac{\partial \text{RSS}}{\partial \tilde{a}_n} \frac{\partial  \tilde{a}_n}{\partial a_{nk}} \\
&= \frac{\left( \sum_{k''=1}^K a_{nk''} \right) \tilde{g}^{(A)}_{nk} - \left( \tilde{g}^{(A)}_n \right)^T a_n}{\left( \sum_{k''=1}^K a_{nk''} \right)^2} \\
&= \frac{\left( \sum_{k''=1}^K a_{nk''} \right) \tilde{g}^{(A)}_{nk} - \sum_{k''=1}^K  \tilde{g}^{(A)}_{n k^{\prime \prime}} a_{nk''} }{\left( \sum_{k''=1}^K a_{nk''} \right)^2} \\
\end{aligned}
\end{equation}
IF we additionally assume that $a_n$ has been l1 normalized in the previous iteration we get 
\begin{equation}
\begin{aligned}
\frac{\partial \text{RSS}}{\partial a_{nk}} 
&= \tilde{g}^{(A)}_{nk} - \sum_{k''=1}^K  \tilde{g}^{(A)}_{n k^{\prime \prime}} a_{nk''} \\
\end{aligned}
\end{equation}
which is exactly the same as in Section 2.2. of Mørup \& Hansen (2012) \autocite{morupArchetypalAnalysisMachine2012}

To write down the algorithm we define $P_{\Sigma_M}$, a function that projects the rows of any matrix $\mathbf{H} \in \mathbb{R}^{N \times M}$ onto the $M$ simplex
\begin{equation}
\begin{aligned}
\tilde{\mathbf{H}} &= P_{\Sigma_M} \left(\mathbf{H} \right) \quad \text{with}\\
\tilde{\mathbf{H}}_{nm} &= \frac{\max(\mathbf{H}_{nm},0)}{\sum_{m^\prime=1}^M \max(\mathbf{H}_{nm^\prime},0)}
\end{aligned}
\end{equation}
Putting everything together, the algorithm in matrix notation is shown in Algorithm~\ref{alg:pcha}

\begin{algorithm}
\caption{Principal Convex Hull Algorithm (PCHA)}
\label{alg:pcha}
\begin{algorithmic}[1]
\State Initialize $\tilde{\mathbf{A}}$, $\tilde{\mathbf{B}}$
\State Initialize $\mu_\mathbf{A} \gets 1$, $\mu_\mathbf{B} \gets 1$
  
\While{not converged or maximum number of iterations is reached}
    \State \fbox{\parbox{0.95\linewidth}{Update $\mathbf{A}$ using projected gradient descent}}
    \State $\mathbf{Z} \gets \tilde{\mathbf{B}} \mathbf{X}$
    \State $\text{RSS}_{\text{old}} \gets \| \mathbf{X} - \mathbf{A} \mathbf{Z} \|_F^2$

    \For{$t = 1$ to $T$}
        \State $\tilde{\mathbf{G}}^{(\mathbf{A})} \gets 2 \left( \tilde{\mathbf{A}} \mathbf{Z} \mathbf{Z}^T - \mathbf{X} \mathbf{Z}^T \right)$
        \State $\mathbf{G}^{(\mathbf{A})} \gets \tilde{\mathbf{G}}^{(\mathbf{A})} - \left(\tilde{\mathbf{G}}^{(\mathbf{A})} \odot \mathbf{A}\right) \mathbf{1}_K \mathbf{1}_K^T$
        
        \For{$j = 1$ to $100T$} \Comment{line search}
            \State $\mathbf{A} \gets \mathbf{A} - \mu_\mathbf{A} \mathbf{G}^{(\mathbf{A})}$
            \State $\tilde{\mathbf{A}} \gets P_{\Sigma_K}(\mathbf{A})$
            \State $\text{RSS}_{\text{new}} \gets \| \mathbf{X} - \tilde{\mathbf{A}} \mathbf{Z} \|_F^2$
            \If{$\text{RSS}_{\text{new}} < \text{RSS}_{\text{old}} + (1+\epsilon)$}
                \State $\mu_\mathbf{A} \gets 1.2 \cdot \mu_\mathbf{A}$
                \State \textbf{break}
            \Else
                \State $\mu_\mathbf{A} \gets 0.5 \cdot \mu_\mathbf{A}$
            \EndIf
        \EndFor
    \EndFor

    \State \fbox{\parbox{0.95\linewidth}{Update $\mathbf{B}$ using projected gradient descent}}
    \State $\text{RSS}_{\text{old}} \gets \| \mathbf{X} - \mathbf{A} \mathbf{B} \mathbf{X} \|_F^2$

    \For{$t = 1$ to $T$}
        \State $\tilde{\mathbf{G}}^{(\mathbf{B})} \gets 2 \left( \tilde{\mathbf{A}}^T \tilde{\mathbf{A}} \tilde{\mathbf{B}} \mathbf{X} \mathbf{X}^T - \tilde{\mathbf{A}}^T \mathbf{X} \mathbf{X}^T \right)$
        \State $\mathbf{G}^{(\mathbf{B})} \gets \tilde{\mathbf{G}}^{(\mathbf{B})} - \left( \tilde{\mathbf{G}}^{(\mathbf{B})} \odot \mathbf{B} \right) \mathbf{1}_N \mathbf{1}_N^T$
        
        \For{$j = 1$ to $100T$} \Comment{line search}
            \State $\mathbf{B} \gets \mathbf{B} - \mu_\mathbf{B} \mathbf{G}^{(\mathbf{B})}$
            \State $\tilde{\mathbf{B}} \gets P_{\Sigma_N}(\mathbf{B})$
            \State $\text{RSS}_{\text{new}} \gets \| \mathbf{X} - \tilde{\mathbf{A}} \tilde{\mathbf{B}} \mathbf{X} \|_F^2$
            \If{$\text{RSS}_{\text{new}} < \text{RSS}_{\text{old}} + (1+\epsilon)$}
                \State $\mu_\mathbf{B} \gets 1.2 \cdot \mu_\mathbf{B}$
                \State \textbf{break}
            \Else
                \State $\mu_\mathbf{B} \gets 0.5 \cdot \mu_\mathbf{B}$
            \EndIf
        \EndFor
    \EndFor

    \State \fbox{\parbox{0.95\linewidth}{Check for Convergence}}
    \State $\mathbf{Z} \gets \tilde{\mathbf{B}} \mathbf{X}$
    \State $\text{RSS} \gets \| \mathbf{X} - \tilde{\mathbf{A}} \mathbf{Z} \|_F^2$
    \If{RSS reduction is sufficient}
        \State \textbf{break}
    \EndIf
\EndWhile
\State \Return $\tilde{\mathbf{A}}, \tilde{\mathbf{B}}, \mathbf{Z}$
\end{algorithmic}
\end{algorithm}

\subsection{Frank-Wolfe Algorithm}

The idea of the Frank-Wolfe algorithm for archetypal analysis is to use gradient information, but to avoid the costly projection step of the PCHA. 

As described above, the objective is convex in $\mathbf{A}$ when fixing $\mathbf{B}$ and vice versa. Furthermore, in this alternating optimization setting, the rows of $\mathbf{A}$ and $\mathbf{B}$ are constrained to the $\Sigma_K$ and $\Sigma_N$ simplex, respectively, which are convex sets. Thus, we have a convex minimization problem over a convex set which can be tackled using the efficient Frank-Wolfe algorithm \autocite{clarksonCoresetsSparseGreedy2010}

\section{Initialization}

\subsection{Furthest Sum}

\section{Weighted Archetypal Analysis}

TODO

\section{Dataset Size Reduction Methods}

\subsection{Coresets}

TODO

\section{References}

\printbibliography[heading=none]

\section{Appendix}

\subsection{Notation}

\begin{itemize}
    \item $N \in \mathbb{N}$ is the number of samples
    \item $D \in \mathbb{N}$ is the number of dimensions
    \item $K \leq \min(N, D)$ is the number of archetypes
    \item $\mathcal{X}=\{\mathbf{x}_1, \ldots, \mathbf{x}_N\}_{n=1}^N$ is our dataset, where each $\mathbf{x}_n \in \mathbb{R}^D$
    \item $\mathbf{X} \in \mathbb{R}^{N \times D}$ is our data matrix where each row is one sample
    \item $\mathbf{Z} \in \mathbb{R}^{K \times D}$ is our matrix of archetypes where each row is one archetype
    \item $\mathcal{Z}=\{\mathbf{z}_1, \ldots, \mathbf{z}_K\}_{k=1}^K$ is set of archetypes, where each $\mathbf{z}_k \in \mathbb{R}^D$
\end{itemize}

\subsection{Algorithms}

\begin{algorithm}
\caption{Principal Convex Hull Algorithm (PCHA)}
\label{alg:pcha_short}
\begin{algorithmic}[1]
\Require Data matrix $\mathbf{X} \in \mathbb{R}^{N \times D}$, learning rates $\mu_{\mathbf{A}} > 0$, $\mu_{\mathbf{B}} > 0$
\State Initialize $\mathbf{A}, \mathbf{B}$
\State $\text{RSS}_{\text{old}} \gets \| \mathbf{X} - \mathbf{A} \mathbf{B} \mathbf{X} \|_F^2$
\While{not converged}
    \Statex \hspace{-\algorithmicindent} \textbf{Update A coefficients:}
    \State $\mathbf{Z} \gets \mathbf{B} \mathbf{X}$ \Comment{compute archetypes}
    \State $\mathbf{G}^{(\mathbf{A})} \gets 2(\mathbf{A} \mathbf{Z} \mathbf{Z}^T - \mathbf{X} \mathbf{Z}^T)$ \Comment{gradient of RSS w.r.t. $\mathbf{A}$}
    \State $\mathbf{A} \gets \mathbf{A} - \mu_{\mathbf{A}} \mathbf{G}^{(\mathbf{A})}$ \Comment{gradient descent step}
    \State $\mathbf{A} \gets P_{\Sigma_K}(\mathbf{A})$ \Comment{project rows of $\mathbf{A}$ onto $K$-simplex}
    \Statex \hspace{-\algorithmicindent} \textbf{Update B coefficients:}
    \State $\mathbf{G}^{(\mathbf{B})} \gets 2(\mathbf{A}^T \mathbf{A} \mathbf{B} \mathbf{X} \mathbf{X}^T - \mathbf{A}^T \mathbf{X} \mathbf{X}^T)$ \Comment{gradient of RSS w.r.t. $\mathbf{B}$}
    \State $\mathbf{B} \gets \mathbf{B} - \mu_{\mathbf{B}} \mathbf{G}^{(\mathbf{B})}$ \Comment{gradient descent step}
    \State $\mathbf{B} \gets P_{\Sigma_N}(\mathbf{B})$ \Comment{project rows of $\mathbf{B}$ onto $N$-simplex}
    \Statex \hspace{-\algorithmicindent} \textbf{Check convergence:}
    \State $\text{RSS}_{\text{new}} \gets \| \mathbf{X} - \mathbf{A} \mathbf{B} \mathbf{X} \|_F^2$
    \State $\text{rel\_decrease} \gets \frac{\text{RSS}_{\text{old}} - \text{RSS}_{\text{new}}}{\text{RSS}_{\text{old}}}$ \Comment{relative decrease in RSS}
    \If{$\text{rel\_decrease} < \epsilon$}
        \State \textbf{break} \Comment{convergence criterion met}
    \EndIf
    \State $\text{RSS}_{\text{old}} \gets \text{RSS}_{\text{new}}$
\EndWhile
\State \Return $\mathbf{A}, \mathbf{B}, \mathbf{Z}$
\end{algorithmic}
\end{algorithm}

\end{document}